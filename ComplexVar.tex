\documentclass{article}
\usepackage{stix}
%\usepackage{latexsym}
%\usepackage{amsmath,amsthm,amssymb,amsfonts}
\usepackage[mathscr]{eucal}
\usepackage{siunitx}
\usepackage{esint}
\usepackage{enumerate}
\usepackage{graphicx}
\usepackage[utf8]{inputenc}

\newcommand{\reals}{\mathbb{R}}
\newcommand{\complex}{\mathbb{C}}

\begin{document}
\title{Complex Variables}
\author{Dr. Francisco Javier Alvarado Chacón}
\maketitle
\section{Introduction}

Throughout history, several sets of numbers have been developed in order to satisfy the current necessities of the time.

First of all, the numbers we use for counting and adding quantities, the Natural numbers
\[
\mathbb{N}=\left\lbrace 1,2,3,\dots\right\rbrace.
\]
Then, when we needed to subtract quantities, it was necessary to use negative numbers, so we have the Integers,
\[
\mathbb{Z}=\left\lbrace\dots,-3,-2,-1,0,1,2,3,\dots\right\rbrace.
\]
Fractions were next in line, when people found themselves in the need to divide integers. These are the Rational numbers
\[
\mathbb{Q}=\left\lbrace p/q \mid p,q\in\mathbb{Z},\,q\ne 0\right\rbrace.
\]
Finally, in the eighteenth and nineteenth centuries, mathematicians like Weierstrass, Dedekind, Cauchy, and Riemann developed the theory of mathematical analysis, based on the set of Real numbers, which we are not going to define here, we will just take for granted that everybody knows them.

This set, denoted by $\reals$, is closed under the operations of addition, subtraction, multiplication, division and exponentiation, but not under the extraction of roots. Extracting a root of a number is equivalent to solve a polynomial equation, but even an equation as simple as $x^2+1=0$ does not have a solution within the real numbers.

In order for this, and other polynomial equations like it, to have a solution, it is necessary to introduce a new set of numbers. Such a set is the set of complex numbers, $\complex$, a superset of $\reals$, where all polynomial equations have a solution. 

We will introduce complex numbers as both, geometric and algebraic objects. Then, we will define the basic arithmetic operations, and we will study the properties of complex functions.

\end{document}
